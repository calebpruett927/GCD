% chapters/13_episteme.tex — Polished for publication
\chapter{Episteme Construction}
\label{ch:episteme}

An episteme is not just a body of results; it is a discipline that \emph{builds itself}. The kernel (identities), closures (thresholds/policies), and narratives (interpretations) are kept distinct; changes are \emph{welds} that preserve \(\kappa\)-continuity on a fixed anchor and leave an auditable trail. This chapter formalizes that discipline, specifies tiered invariance, and sets out \gls{ulrc} governance for artifacts—including this book.

\section{The Discipline That Builds Itself}
\label{sec:discipline}

\begin{definition}[Three layers]
\emph{Identities} are invariant equations and typed outcomes (kernel). \emph{Closures} are selectable thresholds, face policies, and procedures. \emph{Narratives} are interpretations (RCFT or otherwise). Identities never depend on closures; narratives never override identities.
\end{definition}

\begin{definition}[Governed change = weld]
A \emph{weld} is a revision that (i) declares an anchor, (ii) runs pre/post contract-tests on the same anchor, (iii) enforces \(\kappa\)-continuity (secant update; residual published), and (iv) records a \texttt{weld\_id} and bundle manifest hash.
\end{definition}

\begin{proposition}[Budget closure as membership test]
A claim belongs to the discipline iff its audited interval closes
\[
\Delta\kappa \;=\; R\tau_{R} - (D_{\omega}+D_{C})
\quad\text{with residual }\le 10^{-12},
\]
under the declared contract and typed outcomes. Otherwise it is a hypothesis awaiting audit.
\end{proposition}

% ——————————————————————————————————————————————————————————
% REPLACE the existing template eqbox with this filled block.
\begin{eqbox}[Discipline audit—\texttt{book@v1.0} (this volume)]
\small

% Local tweaks so long hex strings can wrap neatly
\begingroup
\def\UrlBreaks{\do\0\do\1\do\2\do\3\do\4\do\5\do\6\do\7\do\8\do\9%
  \do\a\do\b\do\c\do\d\do\e\do\f\do\A\do\B\do\C\do\D\do\E\do\F}
\def\UrlFont{\ttfamily\small}
\setlength{\tabcolsep}{6pt}

\begin{tabularx}{\linewidth}{@{}>{\bfseries}l >{\ttfamily\small\arraybackslash}X@{}}
Layer            & identity | closure | narrative \\
Anchor           & MANIFEST.json (bundle); root\_hash \nolinkurl{7247553fb9576436b097cc0f1e24f5194b816a516a349d3f49775007458cc84a} \\
Contract         & (a,b,\(\epsilon\),p,\(\alpha\))=(-3.7280384,\,10.856631,\,1e-8,\,3,\,1.0) \\
Face policy      & post\_clip+guard (\(\epsilon=10^{-8}\)) \\
Ledger           & \(\Delta\kappa,\,D_\omega,\,D_C,\,R\tau_R,\,\min\mathrm{IC},\,\tau_R\) — (not applicable to release page) \\
Residual         & — (release governance record; budgets close in their chapters) \\
Weld ID          & W-2025-09-23-book-01 \\
Manifest         & \nolinkurl{7247553fb9576436b097cc0f1e24f5194b816a516a349d3f49775007458cc84a} \\
\end{tabularx}
\endgroup

\vspace{0.2\baselineskip}
\raggedright\footnotesize
Typed outcomes (\(\bot\!\mathrm{oor}\), \(\infty_{\mathrm{rec}}\)) are surfaced in budgeted slices; the release record itself is non-numeric and binds governance keys.
\end{eqbox}

% ——————————————————————————————————————————————————————————


\section{Tiered Invariance}
\label{sec:tip}

\begin{definition}[Tiered Invariance Principle (TIP)]
Across tiers (kernel \(\rightarrow\) audits \(\rightarrow\) applications), invariants must be preserved under upgrades:
\begin{enumerate}[leftmargin=1.25em]
  \item \textbf{Kernel invariance.} Identities and typed outcomes are versioned and immutable across releases. If a correction is needed, it is a weld that \emph{reduces} narrative scope, not a silent replacement.
  \item \textbf{Closure portability.} Thresholds and policies may change \emph{by weld}; pre/post tests must show regime-label continuity on the frozen anchor unless the weld expressly re-labels with justification.
  \item \textbf{Caption discipline.} Every artifact states: contract keys \((a,b,\varepsilon,p,\alpha)\), face policy, residual, regime, \texttt{weld\_id}, and manifest hash.
\end{enumerate}
\end{definition}

\begin{remark}[Worst-of joins; typed edges]
Multi-channel joins are worst-of; ties break toward larger \(\mathrm{IC}\). Near-wall pivots and division-by-zero are typed (\(\bot\!\mathrm{oor}\), \(\infty_{\mathrm{rec}}\)) and \emph{never} leak into numeric flows.
\end{remark}

\section{ULRC Governance}
\label{sec:ulrc}

The \gls{ulrc} (\emph{Unified Language of Recursive Collapse}) is the naming/contract layer that keeps symbols stable across domains.

\begin{eqbox}[ULRC registry entry (schema)]
\textbf{Name} — canonical symbol (e.g., \(\omega\), \(C\), \(\tau_R\), \(\kappa\)).\\
\textbf{Type} — scalar \textbar{} channel \textbar{} typed-outcome.\\
\textbf{Definition} — kernel-level equation or typed rule.\\
\textbf{Units/normalization} — affine normalization \(y=(x-a)/b,\ b>0\); declare \((a,b)\).\\
\textbf{Face policy} — \texttt{pre\_clip} by default; pivot to \texttt{post\_clip+guard} near walls (declare \(\varepsilon\)).\\
\textbf{Predicates} — Stable/Watch/Collapse thresholds; Critical overlay for \(\mathrm{IC}<\ICCritical\).\\
\textbf{Caption keys} — contract, residual, regime, \texttt{weld\_id}, manifest hash.\\
\textbf{Change process} — any modification is a \emph{weld} against prior registry state; diffs and tests published.
\end{eqbox}

\begin{eqbox}[Registry weld—\texttt{ulrc://<name>@<ver+1>} (template)]
\small
\begin{tabularx}{\linewidth}{@{}>{\bfseries}l >{\ttfamily}X@{}}
Prior entry    & ulrc://<name>@<ver>; manifest <sha256> \\
Change         & <definition/threshold/predicate delta> \\
Anchor         & <frozen fixtures used to test continuity> \\
Tests          & First-Law closure; regime continuity (or justified re-label) \\
Residual       & <0.0000…> (\(\le 10^{-12}\)) \\
Weld ID        & W-<YYYY-MM>-ULRC-<-me>-<ver+1> \\
Manifest       & <sha256 of new registry bundle> \\
\end{tabularx}
\end{eqbox}

\begin{remark}[Language safety]
Abbreviations and short forms are permitted only if they pass semantic compression tolerance tests (no \(\omega\uparrow\), no \(C\uparrow\), no \(\tau_R\uparrow\) relative to the canonical; see Chapter~\ref{ch:legal}).
\end{remark}

\section{This Book as a Welded Artifact}
\label{sec:book-weld}

This volume obeys the same governance it advocates. Releases are welds; each has pre/post contract-tests, a residual report, and a manifest hash over the project bundle (text, figures, local data).

\subsection*{Release weld IDs}
% ——————————————————————————————————————————————————————————
\begin{table}[h]
  \centering
  \caption{Release ledger for this book (same-anchor welds; bundle manifest printed).}
  \label{tab:book-welds}
  \begingroup
  \footnotesize
  \setlength{\tabcolsep}{3.5pt}
  \renewcommand{\arraystretch}{1.06}
  % Allow line breaks inside long hex strings via hyperref's URL breaker (local to this group)
  \def\UrlBreaks{\do\0\do\1\do\2\do\3\do\4\do\5\do\6\do\7\do\8\do\9%
    \do\a\do\b\do\c\do\d\do\e\do\f}
  \begin{tabularx}{\linewidth}{@{} l l c >{\raggedright\arraybackslash}X >{\ttfamily\footnotesize\raggedright\arraybackslash}X @{}}
    \toprule
    Release & \texttt{weld\_id} & Date (America/Chicago) & Scope note & Manifest root\_hash \\
    \midrule
    v1.0 (first edition)
      & \texttt{W-2025-09-23-book-01}
      & 2025-09-23
      & Kernel finalized; First-Law captions standardized.
      & \nolinkurl{7247553fb9576436b097cc0f1e24f5194b816a516a349d3f49775007458cc84a} \\
    \bottomrule
  \end{tabularx}
  \endgroup
\end{table}



\begin{eqbox}[Release note — v1.0 (first edition)]
\small
\begin{tabularx}{\linewidth}{@{}>{\bfseries}l >{\ttfamily}X@{}}
\textbf{Weld ID:}            & W-2025-09-23-book-01 \\
\textbf{Build date:}         & 2025-09-25 (America/Chicago) \\
\textbf{Manifest root hash:} & 7247553fb9576436b097cc0f1e24f5194b816a516a34\\
                             & 9d3f49775007458cc84a \\
\textbf{Scope:}              & Kernel finalized; First-Law captions standardized. \\
\textbf{Reproducibility:}    & Single-source build; fixed package snapshot; First-Law residuals $\le 10^{-12}$ on audited cards.\\
\end{tabularx}
\end{eqbox}


\subsection*{Reproducibility envelope}
\begin{eqbox}[Reproducibility (declared)]
\textbf{Single-source build.} One project bundle: text, figures, local data; no external services.\\
\textbf{Determinism.} Fixed package snapshot; no shell-escape; identical inputs yield identical PDFs.\\
\textbf{Timestamps.} Absolute dates; America/Chicago for all rendered times.\\
\textbf{Manifests.} Each release ships a SHA-256 over the bundle; captions carry the hash and \texttt{weld\_id}.\\
\textbf{Tests.} Chapter~\ref{ch:proofs} contract-tests pass; sample Budget Reports close with residuals \(\le 10^{-12}\).
\end{eqbox}
% ——————————————————————————————————————————————————————————
% ADD near the Reproducibility subsection
\begin{eqbox}[Manifest summary (recompute key)]
\small
\begin{tabularx}{\linewidth}{@{}>{\bfseries}l >{\ttfamily}X@{}}
Contract         & (a,b,\(\epsilon\),p,\(\alpha\))=(-3.7280384,\,10.856631,\,1e-8,\,3,\,1.0) \\
Face policy      & post\_clip+guard (\(\epsilon=10^{-8}\)) \\
Root hash        & 7247553fb9576436b097cc0f1e24f5194b816a516a349d3f49775007458cc84a \\
Recipe           & concat lines \texttt{path\textbackslash0sha256\textbackslash n} in \emph{sorted} path order \(\rightarrow\) SHA-256 \\
\end{tabularx}
\end{eqbox}
% ——————————————————————————————————————————————————————————

\section{Community Contributions, Versioning, Errata}
\label{sec:community}

\subsection*{Contributions}
External patches are treated as weld proposals:
\begin{itemize}[leftmargin=1.25em]
  \item state the anchor and affected artifacts;
  \item provide pre/post contract-tests on the same anchor;
  \item include at least one Budget Report row demonstrating closure;
  \item propose updated captions (contract keys, regime, \texttt{weld\_id}, manifest hash).
\end{itemize}

\begin{eqbox}[Contribution checklist (accept/reject gate)]
\small
\begin{enumerate}[leftmargin=1.25em]
  \item Anchor declared and reproducible (hash/manifest provided).
  \item Contract keys and face policy listed exactly.
  \item First-Law residual \(\le 10^{-12}\); typed outcomes surfaced if any.
  \item Regime labels consistent with Chapter~\ref{ch:regimes} (or re-label justified).
  \item Weld record includes \texttt{weld\_id} and updated manifest.
\end{enumerate}
\end{eqbox}

\subsection*{Versioning}
Semantic versioning applies to \emph{closures and narratives}. Kernel changes increment a major version and \emph{must} include a migration note that preserves \(\kappa\)-continuity on designated anchors (or declares scope reduction with justification).

\subsection*{Errata as welds}
Errata are not casual notes; they are governed revisions with a ledger row.

\begin{table}[h]
  \centering
  \caption{Errata ledger (same-anchor tests).}
  \label{tab:errata}
  \footnotesize
  \begingroup
  \setlength{\tabcolsep}{4pt}
  \begin{tabularx}{\linewidth}{@{} l l l >{\raggedright\arraybackslash}X @{}}
    \toprule
    \texttt{weld\_id} & Date (America/Chicago) & Artifact & Note (residual) \\
    \midrule
    \texttt{W-2025-09-23-err-01} & 2025-09-23 & Ch.~2, Table 2.1 & Header label fixed; identity unchanged (0.0000). \\
    \texttt{W-2025-09-23-err-02} & 2025-09-23 & Ch.~5, caption   & Tolerance digit corrected; budget re-closed (0.0000). \\
    \bottomrule
  \end{tabularx}
  \endgroup
\end{table}

\section{Closing: A Discipline You Can Join}
\label{sec:closing}

The method is portable: freeze the contract, instrument invariants, close budgets, and weld changes on the same anchor. The governance here—TIP, \gls{ulrc}, captions with residuals and manifests—keeps results legible across domains. To contribute, add a welded artifact with its ledger row and pass the tests. The episteme will accept it, precisely because it can \emph{return}.
