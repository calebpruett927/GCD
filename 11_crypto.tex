% chapters/11_crypto.tex — Polished for publication (standard cards, compile-safe)
\chapter{Collapse Cryptography \& Protocols}
\label{ch:crypto}

Codecs and key-handling protocols are treated as \emph{constructor fields}: their job is to preserve return paths under compression and encryption. The kernel is unchanged: we instrument \(\{\omega,F,S,C,\tau_{R},\mathrm{IC},\kappa\}\), close the First-Law budget, and weld revisions with pre/post tests on the \emph{same anchor}.

\section{Anchor \& Scope (cryptographic runs)}
\label{sec:crypto-anchor}

% ——————————————————————————————————————————————————————————
% Filled anchor for the crypto runs used in this chapter
\begin{eqbox}[Codec/Protocol anchor (publishable form)]
\small

% Local scope: allow neat line breaks in the long hex
\begingroup
\def\UrlBreaks{\do\0\do\1\do\2\do\3\do\4\do\5\do\6\do\7\do\8\do\9%
  \do\a\do\b\do\c\do\d\do\e\do\f\do\A\do\B\do\C\do\D\do\E\do\F}
\def\UrlFont{\ttfamily\small}
\setlength{\tabcolsep}{6pt}

\begin{tabularx}{\linewidth}{@{} >{\bfseries}l >{\raggedright\arraybackslash}X @{}}
Message domain & msgs/v1; AAD: domain=crypto-demo, channel=CRYPTO-01, run=2025-09-23 \\
Framing        & len32 \textbar{} ver=1 \textbar{} nonce=12B; parse policy: strict \\
Algorithm(s)   & aead://aes-256-gcm-siv@v1.0 \\
Nonce policy   & deterministic per-context (96-bit); replay window: 0 \\
Key pipeline   & KDF=HKDF-SHA256@v1, salt=\nolinkurl{7247553fb9576436b097cc0f1e24f5194b816a516a349d3f49775007458cc84a}, rotation=none \\
Contract       & $(a,b,\varepsilon,p,\alpha)=(-3.7280384,\,10.856631,\,10^{-8},\,3,\,1.0)$ \\
Face policy    & \texttt{post\_clip+guard} (\(\varepsilon=10^{-8}\)) \\
\end{tabularx}
\endgroup

\vspace{0.2\baselineskip}
\raggedright\footnotesize
All audits referencing this run must use this anchor verbatim; mixed anchors invalidate the ledger.
\end{eqbox}

% ——————————————————————————————————————————————————————————


\section{Integrity-Preserving Codecs}
\label{sec:crypto-codecs}

A codec maps messages \(m\) to codewords \(c=\mathrm{enc}(m;k)\) and back \(m'=\mathrm{dec}(c;k)\) under key \(k\). The encode–decode cycle is audited as a \emph{return} to the anchor \(m\).

\begin{definition}[Integrity-preserving codec]
A codec is integrity-preserving on a slice if the audited cycle satisfies
\[
\Delta\kappa_{\text{cycle}} \;=\; R\,\tau_{R} \;-\; (D_{\omega}+D_{C}) \;\approx\; 0,
\qquad \text{residual }\le 10^{-12},
\]
under the same anchor and contract. Intuition: the work-like credit of a successful return pays the drift/geometry charges of the transform and its inverse.
\end{definition}

\paragraph{Practical levers.}
\begin{itemize}[leftmargin=1.25em]
  \item \textbf{Drift \(\omega\):} stabilize with deterministic mapping policies (domain separation, nonce discipline, fixed codebooks during a slice).
  \item \textbf{Curvature \(C\):} reduce via authenticated framing (length/version), strict parsing, and redundancy (MAC/tags, checksums).
  \item \textbf{Return delay \(\tau_R\):} shorten with fast verification paths (tag-before-decrypt, fail-fast framing) without raising brittleness.
\end{itemize}

% ——————————————————————————————————————————————————————————
% Standard audit form (no placeholders; compiles as-is)
\begin{eqbox}[Codec audit — \texttt{CRYPTO-01} (standard form)]
\small

% Local scope: allow line breaks in hex
\begingroup
\def\UrlBreaks{\do\0\do\1\do\2\do\3\do\4\do\5\do\6\do\7\do\8\do\9%
  \do\a\do\b\do\c\do\d\do\e\do\f\do\A\do\B\do\C\do\D\do\E\do\F}
\def\UrlFont{\ttfamily\small}
\setlength{\tabcolsep}{6pt}

\begin{tabularx}{\linewidth}{@{} >{\bfseries}l >{\ttfamily\small\raggedright\arraybackslash}X @{}}
Purpose   & Encode–decode cycle on same anchor (Section~\ref{sec:crypto-vignette}) \\
Contract  & (\(a,b,\epsilon,p,\alpha\)) \(=\) \((-3.7280384, 10.856631, 10^{-8}, 3, 1.0)\); face=post\_clip+guard \\
Weld ID   & W-2025-09-23-CRYPTO-01 \\
Manifest  & \nolinkurl{7247553fb9576436b097cc0f1e24f5194b816a516a349d3f49775007458cc84a} \\
Ledger    & \(\Delta\kappa,\,D_{\omega},\,D_{C},\,R\tau_{R},\,\min\mathrm{IC},\,\tau_{R}\) (see rows below) \\
Residual  & \(\le 10^{-12}\) (both rows) \\
\end{tabularx}
\endgroup

\end{eqbox}

% ——————————————————————————————————————————————————————————


\section{Return-Path Ambiguity}
\label{sec:crypto-ambiguity}

Ambiguity means multiple plausible preimages \(m\) for a received \(c\) under the decode policy. Ambiguity raises curvature and can push the ledger negative.

\begin{eqbox}[Ambiguity charge \& mitigations]
Let \(\mathcal{M}(c)=\{m_1,\dots,m_K\}\) be candidates. If tie-breaking depends on unstable context,
\[
\Delta\kappa \;\approx\; -\frac{\alpha}{1+\tau_{R}}\;\Delta C,
\qquad \text{with }\ \Delta C \uparrow\text{ as }K\uparrow.
\]
\textbf{Mitigate} with authenticated framing (length/version), domain separation (context labels are part of the anchor), and explicit policy for unknown/partial frames (\(\bot\!\mathrm{oor}\), not guessing).
\end{eqbox}

Typed outcomes enforce safety: unparseable frames or failed tags are \(\bot\!\mathrm{oor}\) (out-of-domain), never coerced to numbers; division-by-zero in verification paths is \(\infty_{\mathrm{rec}}\). Typed results log and halt the local budget—no leaks.
% ——————————————————————————————————————————————————————————
\begin{eqbox}[Ambiguity policy (publishable)]
\small
\begin{tabularx}{\linewidth}{@{}>{\bfseries}l X@{}}
Framing & length+version authenticated; strict parser; unknown/partial $\rightarrow$ \texttt{\textbackslash bot oor} \\
Domain separation & include context label in AAD; decode is conditioned on it \\
Tie-breaking & deterministic, documented rule; no ambient context \\
Budget effect & ambiguity $\Rightarrow$ $C\uparrow$; ledger debit $\approx \alpha\,\Delta C/(1+\tau_R)$ \\
\end{tabularx}
\end{eqbox}
% ——————————————————————————————————————————————————————————
\section{Key Pipelines (Governed)}
\label{sec:crypto-keys}

Keys participate in the return path. A key pipeline specifies generation, derivation, rotation, and use. We weld pipeline changes like any governed revision.

\begin{eqbox}[Key pipeline contract (slice-wise)]
Declare: key source (KDF inputs/salt), rotation window, nonce policy, and associated-data conventions. Require for each audited slice:
\begin{enumerate}[leftmargin=1.25em]
  \item \textbf{Same anchor:} encode/decode use the same declared key material and metadata policy.
  \item \textbf{Budget closure:} \(\Delta\kappa = R\tau_{R}-(D_{\omega}+D_{C})\) with residual \(\le 10^{-12}\).
  \item \textbf{Weld discipline:} rotations/format changes are welds with pre/post tests on the same anchor; publish \texttt{weld\_id}.
\end{enumerate}
\end{eqbox}

\paragraph{Design heuristics.}
\begin{itemize}[leftmargin=1.25em]
  \item Deterministic, per-context nonces stabilize mapping (\(\omega\downarrow\)).
  \item Authenticated encryption with explicit framing reduces brittleness (\(C\downarrow\)).
  \item Tag-first verification shortens confirmation (\(\tau_{R}\downarrow\)) without raising \(C\).
\end{itemize}

\section{Audited Vignette: Encryption–Decryption Cycle}
\label{sec:crypto-vignette}

We audit a single encode–decode cycle on the same anchor message. Row 1 uses a brittle frame (higher \(C\)) and loose nonce discipline (higher \(\omega\)). Row 2 hardens the codec with authenticated framing and deterministic nonce derivation.

% ——————————————————————————————————————————————————————————
\begin{eqbox}[Budget Report—\texttt{W-2025-09-23-CRYPTO-01} (baseline)]
\small
\begin{tabularx}{\linewidth}{@{}>{\bfseries}l >{\ttfamily}X@{}}
Purpose           & Baseline codec (loose nonce discipline, brittle frame) \\
$\Delta\kappa$    & -0.0040 \\
$D_{\omega}$      & 0.0100 \\
$D_{C}$           & 0.0120 \\
$R\tau_{R}$       & 0.0180 \\
$\min\mathrm{IC}$ & 0.9100 \\
$\tau_{R}$        & 1.8000 \\
Residual          & 0.0000 \\
Weld ID           & W-2025-09-23-CRYPTO-01 \\
Manifest          & 7247553fb9576436b097cc0f1e24f5194b816a516a349d3f49775007458cc84a \\
\end{tabularx}

\vspace{0.2\baselineskip}
\raggedright\footnotesize
Check: $-0.0040 = 0.0180 - (0.0100+0.0120)$; residual $=0.0000\le 10^{-12}$.
\end{eqbox}
% ——————————————————————————————————————————————————————————

% ——————————————————————————————————————————————————————————
\begin{eqbox}[Budget Report—\texttt{W-2025-09-23-CRYPTO-01} (hardened)]
\small
\begin{tabularx}{\linewidth}{@{}>{\bfseries}l >{\ttfamily}X@{}}
Purpose           & Constant-integrity codec (AEAD framing + deterministic nonces) \\
$\Delta\kappa$    & +0.0060 \\
$D_{\omega}$      & 0.0060 \\
$D_{C}$           & 0.0060 \\
$R\tau_{R}$       & 0.0180 \\
$\min\mathrm{IC}$ & 0.9600 \\
$\tau_{R}$        & 1.2000 \\
Residual          & 0.0000 \\
Weld ID           & W-2025-09-23-CRYPTO-01 \\
Manifest          & 7247553fb9576436b097cc0f1e24f5194b816a516a349d3f49775007458cc84a \\
\end{tabularx}

\vspace{0.2\baselineskip}
\raggedright\footnotesize
Check: $+0.0060 = 0.0180 - (0.0060+0.0060)$; residual $=0.0000\le 10^{-12}$.
\end{eqbox}
% ——————————————————————————————————————————————————————————


\paragraph{Interpretation.}
Authenticated framing removes ambiguity (\(C\downarrow\)); deterministic nonces stabilize the mapping (\(\omega\downarrow\)); verification becomes faster (\(\tau_{R}\downarrow\)). The ledger moves from a small net loss to a net gain while maintaining \(\min\mathrm{IC}\) in the Stable band.


\section{Protocol Checklist (publication standard)}
\begin{enumerate}[leftmargin=1.25em]
  \item \textbf{Same anchor.} Declare message domain, associated data, framing version, and nonce/IV policy.
  \item \textbf{Typed outcomes.} Parse failures \(\rightarrow \bot\!\mathrm{oor}\); verification divisions \(\rightarrow \infty_{\mathrm{rec}}\); never coerce to numerics.
  \item \textbf{Caption discipline.} Contract keys \((a,b,\varepsilon,p,\alpha)\), face policy, tolerance, weld ID, manifest hash.
  \item \textbf{Near-wall rule.} If drift spikes (\(\omega\uparrow\)), pivot to \texttt{post\_clip+guard} and log \(\varepsilon\).
  \item \textbf{Closure tests.} First-Law residual \(\le 10^{-12}\) and (when applicable) a finite-difference rate check (Chapter~\ref{ch:proofs}).
\end{enumerate}

\section{Threats \texorpdfstring{$\rightarrow$}{} Invariants (quick map)}
\begin{description}[leftmargin=1.2em,labelindent=0em,style=nextline]
  \item[Nonce reuse / replay] raises \(\omega\) (mapping no longer stable).
  \item[Ambiguous framing / parsing] raises \(C\) (small changes, large effects).
  \item[Slow verification paths] raise \(\tau_{R}\) (late confirmation).
  \item[Aggressive compression] lowers redundancy (\(S\downarrow\)) but often raises \(C\) and \(\omega\) unless codebooks are fixed.
\end{description}

\section{Weld Record (governance)}
\noindent\textbf{Weld ID:} \texttt{W-2025-09-23-CRYPTO-01}. \;
\textbf{Anchor:} same message domain, same AAD and framing version. \;
\textbf{Tests:} First-Law closure and finite-difference rate check on identical-length intervals. \;
\textbf{Residuals:} \(0.0000\) both rows. \;
\textbf{Policy deltas:} added authenticated framing; deterministic nonces; tag-first verification.

\medskip
\noindent
Collapse-aware cryptographic engineering follows the same pattern as every other constructor field: freeze the contract, instrument the invariants, close the budget, and weld governed changes on the same anchor. Only the actuators differ; the ledger does not.
